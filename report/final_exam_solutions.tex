\documentclass[]{article}
\usepackage[a4paper, total={6in, 8in}]{geometry}
\usepackage[english]{babel}	
\usepackage[utf8]{inputenc} % Umlaute
\usepackage{amssymb}
\usepackage{amsmath} 
\usepackage{fancyhdr}
\usepackage{fancyvrb} % Für regex Umgebung
\usepackage{multicol} % Mehrere Spalten
\usepackage{graphicx} % bilder
\usepackage[table]{xcolor}
\usepackage{hyperref}
\pagestyle{fancy}
\fancyhf{} %clears the header and footer
\rhead{Professor Robert Feldmann (lectures) \\ Dr. Darren Reed (exercise classes) \\ Dimakopoulos Vasileios (TA) \\ Elia Cenci (TA)}
\lhead{Universität Zürich \\ Institute for Computational Science \\ Spring Semester 2021 \\ ESC403}
\fancyfoot[C]{\thepage}


\title{\textbf{INTRODUCTION TO DATA SCIENCE FINAL EXAM 2021}}
\author{Solutions from David Linder}

\begin{document}
	\maketitle
	\thispagestyle{fancy}
	\section{Time Series}
	\subsection{}
	To be considered stationary a time series should have the following properties:
	\begin{itemize}
		\item The mean $E[X_t]$ is the same for all times $t$
		\item The variance $Var[X_t]$ is the same for all times $t$
		\item The covariance between $X_t$ and $X_{t-1}$ is the same for all $t, n$
	\end{itemize}
	That means we want
	\begin{itemize}
		\item no obvious trends
		\item constant variance with time
		\item constant autocorrelation structure over time
		\item no periodic fluctuations (no seasonality)
	\end{itemize}
	I found some examples \href{https://otexts.com/fpp2/stationarity.html}{here}: In figure \ref{fig:time_series} we see that most of the series are non-stationary except series (b) and (g). In series (g) there are cycles but they are not periodic. 
	\begin{figure}
		\centering
		\includegraphics[width=0.8\textwidth]{images/time_series.png}
		\caption{(a) Google stock price for 200 consecutive days; (b) Daily change in the Google stock price for 200 consecutive days; (c) Annual number of strikes in the US; (d) Monthly sales of new one-family houses sold in the US; (e) Annual price of a dozen eggs in the US (constant dollars); (f) Monthly total of pigs slaughtered in Victoria, Australia; (g) Annual total of lynx trapped in the McKenzie River district of north-west Canada; (h) Monthly Australian beer production; (i) Monthly Australian electricity production.}
		\label{fig:time_series}
	\end{figure}
	
	\subsection{}
	
	\begin{itemize}
		\item By looking at the data we can clearly see that this time series is not stationary. After log transformed the $X$-data we can see in figure \ref*{fig:rollingmean} that although the standard deviation has a small variation the mean increases over time. Also the results of the Dickey-Fuller test is that we can not reject $H_0$ (TS is non-stationary) because the test statistic value is not less than the critical value. Here is the data in detail: 
		\begin{center}
			\begin{tabular}{|l|r|}
				\hline Test Statistic & $-0.2312$ \\
				\hline p-value & $0.9347$ \\
				\hline Lags & $22.0000$ \\
				\hline Observations & $975.0000$ \\
				\hline Critical Value $(1 \%)$ & $-3.4371$ \\
				\hline Critical Value $(5 \%)$ & $-2.8645$ \\
				\hline Critical Value $(10 \%)$ & $-2.5684$ \\
				\hline
			\end{tabular}
			\label{tab:dickey}
		\end{center}
		I will use the moving average smoothing method which we used in the exercise class to eliminate the trend. First i calculated the rolling mean with the pandas function and then subtracted this from the original series (code: timeseries/1.2.py). Then i dropped the nan values that resulted from averaging over the time window (i tried out different sizes of that window and ended up with a value of 60). In figure \ref{fig:rollingmean_diff} we see the result and that we eliminated the trend. Let's take a look at the test statistics:
		\begin{center}
			\begin{tabular}{|l|r|}
				\hline Test Statistic & $-10.7860$ \\
				\hline p-value & $0.0000$ \\
				\hline Lags & $1.0000$ \\
				\hline Observations & $937.0000$ \\
				\hline Critical Value $(1 \%)$ & $-3.4373$ \\
				\hline Critical Value $(5 \%)$ & $-2.8646$ \\
				\hline Critical Value $(10 \%)$ & $-2.5684$ \\
				\hline
			\end{tabular}
		\end{center}
		The value is lower than all the critical values. Therefore we can say at least with 99\% confidence this is now a stationary time-series.
		
		\begin{figure}
			\centering
			\includegraphics[width=1\textwidth]{images/ts_log_moving_avg_diff.png}
			\caption{The result from smoothing.}
			\label{fig:rollingmean_diff}
		\end{figure}
		\item For answering this question we need to find the p-value. For this value to find we look at the plot of the partial autocorrelation function (PACF). From figure \ref{fig:pacf} we find a p-value of 3. That means that 3 values of at prior times are expected to directly affet a given current value of the time series.
		\begin{figure}
			\centering
			\includegraphics[width=1\textwidth]{images/partialautocorrelation.png}
			\caption{Plot of the partial autocorrelation function. The x-axis are the lags. One can see that the first time the PACF crosses the confidence interval (blue) is at lag 3.}
			\label{fig:pacf}
		\end{figure}
		\item 
		\begin{figure}
			\centering
			\includegraphics[width=1\textwidth]{images/autocorrelation.png}
			\caption{The autocorrelation function. The function crosses the confidence interval at lag 7 $\implies q=7$.}
			\label{fig:acf}
		\end{figure}
	\end{itemize}
	
	\subsection{}
	\begin{itemize}
		\item 
		\begin{figure}
			\centering
			\includegraphics[width=1\textwidth]{images/ts_moving_avg_B.png}
			\caption{Average temperature anomaly raw data.}
			\label{fig:rollingmean_B}
		\end{figure}
		\begin{figure}
			\centering
			\includegraphics[width=1\textwidth]{images/ts_moving_avg_diff_B.png}
			\caption{Average temperature anomaly after differencing.}
			\label{fig:rollingmean_diff_B}
		\end{figure}
		\begin{figure}
			\centering
			\includegraphics[width=1\textwidth]{images/autocorrelation_B.png}
			\caption{The autocorrelation function.}
			\label{fig:acf_B}
		\end{figure}
		\begin{figure}
			\centering
			\includegraphics[width=1\textwidth]{images/partialautocorrelation_B.png}
			\caption{The partial autocorrelation function.}
			\label{fig:pacf_B}
		\end{figure}
		\begin{figure}
			\centering
			\includegraphics[width=1\textwidth]{images/res_dens.png}
			\caption{Residuals and density.}
			\label{fig:res_dens}
		\end{figure}
		\item
		\begin{figure}
			\centering
			\includegraphics[width=1\textwidth]{images/forecast_2010-2021.png}
			\caption{Forecast 2010 - 2021}
			\label{fig:forecast_2010-2021}
		\end{figure}
		\begin{figure}
			\centering
			\includegraphics[width=1\textwidth]{images/forecast_2010-2021_RMSE.png}
			\caption{Forecast 2010 - 2021 RMSE}
			\label{fig:forecast_2010-2021_RMSE}
		\end{figure}
		\item
	\end{itemize}
	\section{Image classification}
	\subsection{Image classification}
	If better means higher accuracy then i would choose a CNN for image classification. RF performs well on categorical data while a CNN handles numerical input very well. Here we have pixels with numerical values between 0 and 255. We can also tune more parameters in a CNN like kind/number of layers, epochs and learning rate. Also different activation functions for the neurons can be chosen. Finally a CNN learns how to apply a filter to an input during training such that certain features in the image can be recognized. A RF can not take advantage of such structures in images.
	
	On the other hand a CNN needs a lot of data to perform well. I don't know exactly if our dataset is large enough but if i compare with other models they used hundred thousands or even millions of images to train a CNN.

	\subsection{Classification performance}
	\begin{itemize}
		\item Accuracy CNN: xx.xx\%, Accuracy RF: xx.xx\%	
		\item The classification error rate is the proportion of instances misclassified divided by the whole set of instances:
		\begin{equation}
			cer = \frac{FP + FN}{TP + TN + FP + FN}
		\end{equation}
		In our case the error rate was: CNN: x, RF: x
		\item Does any model perform better?
		\item The prediction for the unlabled dataset is the following:
	\end{itemize}
	
	\subsection{Hyper parameter tuning}
	
	\subsection{Interpretation of model performance}
	\subsection{Bonus Question}
\end{document}